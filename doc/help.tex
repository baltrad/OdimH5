\documentclass[11p]{article}
\usepackage{eng}

\begin{document}

\setlength{\fboxrule}{1pt}
\setlength{\fboxsep}{9pt}
\begin{flushleft}

\small
Copyright (C) 2009-2013 Institute of Meteorology and Water Management, IMGW\\
\vspace{.5cm}

\huge
\framebox[\textwidth][l]{OdimH5 User's Guide}
\end{flushleft}

\begin{center}
Software version: 2.16 \\
Last update of this manual: \today \\
Authors: Maciej Szewczykowski, Łukasz Wojtas\\
\end{center}

\begin{flushleft}

This document is a user's guide on how to use OdimH5.
\end{flushleft}

\vspace{1cm}
\small \noindent OdimH5 is a console HDF5 exchange software designed to work on
radar data files. It provides XML descriptor handler, HDF5 converter, FTP and
Baltrad feeder mode. \normalsize

\section{Introduction}

\subsection{License}

OdimH5 is free software: you can redistribute it and/or modify it under the
terms of the GNU Lesser General Public License as published by the Free Software
Foundation, either version 3 of the License, or (at your option) any later
version.

OdimH5 is distributed in the hope that it will be useful, but WITHOUT ANY
WARRANTY; without even the implied warranty of MERCHANTABILITY or FITNESS FOR A
PARTICULAR PURPOSE.  See the GNU Lesser General Public License for more details.

You should have received a copy of the GNU Lesser General Public License along
with OdimH5.  If not, see \href{http://www.gnu.org/licenses/}
{http://www.gnu.org/licenses/}.

\subsection{Overview}

The OdimH5 is a Java-based tool working on meteorological radar data. It allows
users to create an XML descriptor which contains all major information require to
create a HDF5 file. The application also allows users to convert specific radar
data to HDF5 format based on the descriptor. For \texttt{Baltrad} users it
provides automatic online mode which feeds \texttt{BaltradDex} with actual
data.

The application was implemented using the Java\textsuperscript{TM}~2 Platform,
which is machine-independent.

\subsubsection*{HDF libraries} 

  This release was built and tested with HDF5-1.8.4 Patch 1 with HDF5~1.6
  compatibility flag. For information on new features in HDF5 Release 1.8.0 and
  format compatibility considerations, please visit\\
  \href{http://www.hdfgroup.org/HDF5/doc/ADGuide/CompatFormat180.html}
  {http://www.hdfgroup.org/HDF5/doc/ADGuide/CompatFormat180.html}.

\subsubsection*{Apache libraires}

This software was built using Apache Commons libraries 
\href{http://commons.apache.org/}{http://commons.apache.org/}.

\subsubsection*{jpathwatch library}

This software was built using jpathwatch 0.94 
\href{http://jpathwatch.wordpress.com/}{http://jpathwatch.wordpress.com/}.

\subsubsection*{Platforms}

This version is available for following operating systems:
\begin{itemize}

  \item 32-bit distribution -- runs on 32-bit or 64-bit systems with 32-bit JRE
  \begin{itemize}
      \item Windows (OdimH5\_win32)
      \item Linux (OdimH5\_linux32)
      
	\end{itemize}
	\item 64-bit distribution -- runs on 64-bit systems with 64-bit JRE

      \begin{itemize}
      \item Windows (OdimH5\_win64)
      \item Linux (OdimH5\_linux64)
      
	\end{itemize}
   
\end{itemize}

\subsection{Supported radar systems}

This version can work with limited radar systems and products listed below.

\subsubsection*{Platforms}
\begin{itemize}
  \item Gematronik RAINBOW
\end{itemize}

\subsubsection*{Type of product}
\begin{itemize}
  \item Polar Volume Scan (ver. 5.2x and 5.3x)
  \begin{itemize}
    \item dBZ - Reflectivity
%    \item uPhiDP - Differential Phase Shift
%    \item KDP - Specific Differential Phase Shift
%    \item RhoHV - Correlation Coefficient
    \item V - Radial Velocity
  \end{itemize}
  \item Cartesian image and composite (ver. 5.2x)
  \begin{itemize}
    \item PPI - Plan Position Indicator
    \item CAPPI - Constant Altitude PPI
    \item MAX - Maximum Display
    \item EHT - Echo Height
    \item SRI - Surface Rainfall Intensity
    \item PAC - Precipitation Accumulation
    \item VIL - Vertical Integrated Liquid
    \item HSHEAR - Horizontal Shear 
  \end{itemize} 
%   \item Vertical profile (Not implemented)
  \item Range-height indicator
  \begin{itemize}
    \item RHI - Range Height Indicator
  \end{itemize} 
\end{itemize} 

\subsection{Features}

\begin{itemize}
    \item Convert a single file to HDF5.
    \item Convert a file from HDF5 to Rainbow format.
    \item Convert and send files via FTP in continueus mode.
    \item Convert and send files to Baltrad Dex in continueus mode.
    
\end{itemize}

\section{Getting Started}

\subsection{Installation and setting}

The newest version of OdimH5 is available on Opera FTP server
(\texttt{ftppro.knmi.nl}). Download version for your operating system and
extract. Alternatively (for Baltrad users only) use \texttt{Git} a distributed
revision control system and clone project from baltrad server. To do this use following command
\begin{kod}
git clone gitosis@git.baltrad.eu:OdimH5.git
\end{kod}

Installation of OdimH5 software is driven by \texttt{Apache Ant} and requires
several options passed to the build script. The following options are required:
\begin{itemize}
    \item \texttt{prefix} -- installation prefix / target directory 
    \item \texttt{hostname} -- name of the host on which the installation is
performed, preferably a domain name  
    \item \texttt{system} -- operating system type, possible choices are 'linux'
or 'win' 
    \item \texttt{arch} -- system architecture, possible choices are '32' and
'64'
\end{itemize}

Example of installation command:
\begin{kod} 
ant -Dprefix=/home/odim -Dhostname=myhost.org -Dsystem=linux -Darch=64 install
\end{kod}

In this case the software will be installed in linux/unix-type system under 
\texttt{/home/odim} directory using a set of 64-bit native libraries.

There is also an alternative installation procedure which install the
application only for converting purposes:
\begin{kod} 
ant -f buildcnvt.xml -Dprefix=/opt -Dsystem=linux -Darch=64 install
\end{kod}

The application reads options from \texttt{options.xml} file stored in
\texttt{conf} folder. The following options can be provided:

\begin{longtable}{|l|p{9cm}|}

\hline

\texttt{<radar>}&Every radar is represented by this element. The attribute
\texttt{name} should be 3-letter radar name, same as one stored in raw volume
file e.g. \texttt{name="SWI"}. This section contains one obligated tag:
\texttt{WMO\_id}, and following extra tags: \texttt{originator}, \texttt{product\_id} (or \texttt{file\_name}),
\texttt{directory} and \texttt{nrays}\\ \hline 
\texttt{<WMO\_id>}& WMO block and station number, other indicators like NOD,
PLC can be also added here seperated with comma\\
\hline \texttt{<product\_id>}&File name prefix compliant with ODIM file naming
convention: \texttt{pflag\_productidentifier\_oflag\_originator\_}. For details
see Opera document WD-2010-02. \\ \hline
\texttt{<originator>}&File name prefix compliant with ODIM file naming
convention: \texttt{pflag\_productidentifier\_oflag\_originator\_}. For details
see Opera document WD-2010-02. \\ \hline
\texttt{<file\_name>}&File name prefix compliant with ODIM file naming
convention: \texttt{pflag\_productidentifier\_oflag\_originator\_}. For details
see Opera document WD-2010-02. \\ \hline
\texttt{<directory>}& Path of the directory where radar volume data are stored.
This directory will be watched in feeder mode by application, and every new
file in the directory will awake the process of conversion. \\ \hline
\texttt{<nrays>}&Number of rays that are allowed, if volume contains more then
that number, extra rays are discard. \\ \hline \hline
%\texttt{<location>}& Full radar name for BALTRAD DEX. If not provided Odim will
%use name from RAINBOW volume file.\\ \hline
\texttt{<baltrad>}& Section containing BALTRAD options for
\texttt{server}. \\ \hline \texttt{<server>}& Address of HTTP server.\\ \hline
\hline
%\texttt{<sender>}& Sender name.\\ \hline
\texttt{<ftp>}& Section containing FTP settings, obligatory are radar names,
address, login and password, optional are remote directory, subfolders\\
\hline \texttt{<radars>}& List of radars to be sent using FTP. Use radar names 
separeted by space (names should be same as provided in \texttt{<radar
name=XXX>}).\\
\hline \texttt{<address>}& Address of FTP server. Use localhost for local
copies.\\
\hline \texttt{<login>}& Login to FTP server. \\ 
\hline \texttt{<password>}& Password to FTP server. \\
\hline \texttt{<directory>}& Remote directory. \\
\hline \texttt{<subfolders>}& Provide this option if subfolders for different
radars and date should be created on remote server. Available options are: 0 -
no subfolders, 1 - subfolders for different radars (eg. \texttt{POZ/}), radars
name are the same as provided in radar section, 2 - subfolders for different radars and
date (\texttt{POZ/2013-04-03})
\\
\hline

\end{longtable}

Example \texttt{options.xml} file:

\begin{kod}
<?xml version="1.0" ?>\\
<!-- options -->\\
<options>\\
\hspace*{0.5cm}<radar name="BRZ">\\
\hspace*{1cm}<WMO\_id>WMO:12568,NOD:plbrz,PLC:Brzuchania</WMO\_id>\\
\hspace*{1cm}<originator>SOWR</originator>\\
\hspace*{1cm}<product\_id>48</product\_id>\\
\hspace*{1cm}<directory>/home/volumes/BRZ\_250\_Z.vol</directory>\\
\hspace*{1cm}<nrays>360</nrays>\\
\hspace*{0.5cm}</radar>\\
\hspace*{0.5cm}<baltrad>\\
\hspace*{1cm}<server>http://172.30.9.34:8084/BaltradDex/dispatch.htm</server>\\
% \hspace*{1cm}<sender>Baltrad.IMGW.pl</sender>\\
\hspace*{0.5cm}</baltrad>\\
\end{kod}
\begin{kod}
\hspace*{0.5cm}<ftp>\\
\hspace*{1cm}<radars>BRZ POZ LEG</address>\\
\hspace*{1cm}<address>ftp.address</address>\\
\hspace*{1cm}<login>login</login>\\
\hspace*{1cm}<password>secret</password>\\
\hspace*{1cm}<directory>/radars</directory>\\
\hspace*{1cm}<subfolders>1</subfolders>\\
\hspace*{0.5cm}</ftp>\\
</options>
\end{kod}

\section{Using OdimH5}

OdimH5 converter software provides a set of scripts for convenient data 
processing. The scripts are located in \texttt{bin} directory.
\begin{uwaga}
NOTE: It is not recommended to run odimH5.jar executable directly, unless there
is a good reason to do so. Otherwise use only wrapper scripts.
\end{uwaga}

\subsubsection*{odimH5}
This script start the appliacation, see next section (advenced usage) for
details. Adding \texttt{bin} to the system \texttt{PATH} will make the
application run from any location.

\subsubsection*{createDescriptor.sh} 
Creates descriptor and dataset files
corresponding to a given data file  

\begin{kod}
createDescriptor.sh file descriptor platform object mode
\end{kod}
\begin{itemize}
    \item\texttt{file} -- input file name, either rawdata or product
    \item\texttt{descriptor} -- output descriptor file name
    \item\texttt{platform}  -- type of processing software.
%RAINBOW: Gematronik Rainbow\\
    \item\texttt{object} -- ODIMH5 file object type.
%PVOL: polar volume\\
%CVOL: carthesian volume
%SCAN polar scan\\
%RAY: single polar ray
%AZIM: azimuthal object
%IMAGE: 2-D cartesian image\\
% COMP: cartesian composite image(s)
% XSEC: 2-D vertical cross section(s)
% VP: 1-D vertical profile
% PIC: embedded graphical image
    \item\texttt{mode} -- use v option for verbose mode
\end{itemize}

\subsubsection*{convertFromDescriptor.sh}  
Converts given native data file into hdf5 file using the descriptor

\begin{kod}
convertDescriptor.sh descriptor file mode
\end{kod}
\begin{itemize}
    \item \texttt{descriptor} -- input descriptor file name
    \item \texttt{file} -- output data file name (hdf5 format)
    \item \texttt{mode} -- use v option for verbose mode
\end{itemize}
\subsubsection*{convertNative.sh}
Converts given native data file directly into hdf5 file

\begin{kod}
convertNative.sh file mode
\end{kod}
\begin{itemize}
    \item \texttt{descriptor} -- input descriptor file name
    \item \texttt{file} -- output data file name (hdf5 format)
    \item \texttt{mode} -- use v option for verbose mode
\end{itemize}
\subsubsection*{feedToBaltrad.sh}
Sends given file to BALTRAD node

\begin{kod}
feedToBaltrad.sh input\_file node\_address mode
\end{kod}
\begin{itemize}
    \item \texttt{input\_file} -- input file name (hdf5 format)
    \item \texttt{node\_address} -- BALTRAD node address\\
                        e.g. http://127.0.0.1:8084/BaltradDex/dispatch.htm
    \item \texttt{mode} -- use v option for verbose mode
\end{itemize}

\subsubsection*{startFeeder.sh} 

Runs BALTRAD feeder. BALTRAD feeder watches a given
directory specified in conf/options.xml file. Once a new file is stored
                  in this directory, it is sent to a BALTRAD node with a given
                  address specified on conf/options.xml file. Refer to OdimH5
                  manual for more information.

\begin{kod}
startFeeder.sh mode
\end{kod}
\begin{itemize}
    \item \texttt{mode} -- use v option for verbose mode
\end{itemize}

\section{Advenced usage}

\subsection{Convertion mode}

\texttt{OdimH5} provides two conversion modes. First one creates XML descriptor,
that can be used later to create HDF file. Second mode converts raw data
directly to HDF5.

\subsubsection*{Prepare descriptor}

Descriptor is an XML file, which structure corresponds to HDF5 file. To prepare
descriptor use the following parameters:
\begin{itemize}
  \renewcommand{\labelitemi}{ }
  \item \texttt{-i}\hspace{0.5cm}Input file's path.\\Program can work with
  only one file simultaneously.
  \item \texttt{-o}\hspace{0.5cm}Output file's path.\\It is suggested to use
  \texttt{.xml} filename extension.
  \item \texttt{-p}\hspace{0.5cm}Radar platform's name.\\At the moment only
  Gematronik's RAINBOW software is supported.
  \item \texttt{-f}\hspace{0.5cm}Product format.\\Use one of the formats
  listed above according to input data type.
  \item \texttt{-v}\hspace{0.5cm}Verbose mode.\\It is optional and displays
  status of progress of program work.
\end{itemize}

Example of use:

\begin{kod}
\$ odim -i input.ppi -o ppi.xml -p RAINBOW -f IMAGE -v
\end{kod}

\subsubsection*{Prepare HDF5 file from descriptor}

It requires XML descriptor as an input file. To prepare HDF5 use the following
parameters:

\begin{itemize}
  \renewcommand{\labelitemi}{ }
  \item \texttt{-i}\hspace{0.5cm}Input file's path.\\Program can work with
  only one file simultaneously.
  \item \texttt{-o}\hspace{0.5cm}Output file's path.\\It is suggested to use
  \texttt{.h5} filename extension. If no output file name provided program will
  generate one using ODIM convention, or if no prefix in options.xml were set
  program will use input file name with \texttt{.h5} filename extension.
  \item \texttt{-v}\hspace{0.5cm}Verbose mode.\\It is optional and displays
  status of progress of program work.
\end{itemize}

Example of use:

\begin{kod}
\$ odim -i ppi.xml -o output.h5
\end{kod}

\subsubsection*{Example of descriptor file}

Header:
\begin{kod}
<?xml version="1.0" encoding="UTF-8"?>\\
<!--ODIM\_H5 descriptor file, platform: RAINBOW, file object: PVOL-->\\
\end{kod}
Root group:
\begin{kod}
<group name="/">\\
\end{kod}
What group:
\begin{kod}
    <group name="what">\\
\hspace*{0.5cm}<attribute class="string" name="object">PVOL</attribute>\\
\hspace*{0.5cm}<attribute class="string" name="version">H5rad 2.0</attribute>\\
\hspace*{0.5cm}<attribute class="string" name="date">20101022</attribute>\\
\hspace*{0.5cm}<attribute class="string" name="time">123044</attribute>\\
\hspace*{0.5cm}<attribute class="string" name="source">WMO:12374</attribute>\\
    </group>\\
\end{kod}
Where group:
\begin{kod}
    <group name="where">\\
\hspace*{0.5cm}<attribute class="double" name="lon">20.960630</attribute>\\
\hspace*{0.5cm}<attribute class="double" name="lat">52.405220</attribute>\\
\hspace*{0.5cm}<attribute class="double" name="height">119.000000</attribute>\\
    </group>\\
\end{kod}
How group:
\begin{kod}
    <group name="how">\\
\hspace*{0.5cm}<attribute class="long" name="startepochs">1287743444000</attribute>\\
\hspace*{0.5cm}<attribute class="long" name="endepochs">1287743444000</attribute>\\
\hspace*{0.5cm}<attribute class="string" name="system">GEMA</attribute>\\
\hspace*{0.5cm}<attribute class="string" name="software">RAINBOW</attribute>\\
\hspace*{0.5cm}<attribute class="string" name="sw\_version">5.29.5</attribute>\\
\hspace*{0.5cm}<attribute class="double" name="beamwidth">1</attribute>\\
\hspace*{0.5cm}<attribute class="double" name="wavelength">0.0531</attribute>\\
    </group>\\
\end{kod}
Dataset group:
\begin{kod}
<group name="dataset1">\\
\end{kod}
Dataset children:
\begin{kod}
        <group name="what">\\
\hspace*{0.5cm}<attribute class="string" name="product">SCAN</attribute>\\
\hspace*{0.5cm}<attribute class="string" name="startdate">20101022</attribute>\\
\hspace*{0.5cm}<attribute class="string" name="starttime">123044</attribute>\\
\hspace*{0.5cm}<attribute class="string" name="enddate">20101022</attribute>\\
\hspace*{0.5cm}<attribute class="string" name="endtime">123059</attribute>\\
        </group>\\
\end{kod}
\begin{kod}
        <group name="where">\\
\hspace*{0.5cm}<attribute class="double" name="elangle">0.5</attribute>\\
\hspace*{0.5cm}<attribute class="long" name="nbins">250</attribute>\\
\hspace*{0.5cm}<attribute class="double" name="rstart">0</attribute>\\
\hspace*{0.5cm}<attribute class="double" name="rscale">1000.0</attribute>\\
\hspace*{0.5cm}<attribute class="long" name="nrays">360</attribute>\\
\hspace*{0.5cm}<attribute class="long" name="a1gate">0</attribute>\\
        </group>\\
\end{kod}
\begin{kod}
        <group name="data1">\\
\hspace*{0.5cm}<group name="what">\\
\hspace*{1cm}<attribute class="string" name="quantity">DBZH</attribute>\\
\hspace*{1cm}<attribute class="double" name="gain">0.5</attribute>\\
\hspace*{1cm}<attribute class="double" name="offset">-32.0</attribute>\\
\hspace*{1cm}<attribute class="double" name="nodata">255.0</attribute>\\
\hspace*{1cm}<attribute class="double" name="undetect">0.0</attribute>\\
\hspace*{0.5cm}</group>\\
\hspace*{0.5cm}<dataset CLASS="IMAGE" IMAGE\_VERSION="1.2" chunk="20x20"\\
\hspace*{0.5cm}data\_size="8" data\_type="integer" dimensions="360x250"\\
\hspace*{0.5cm}gzip\_level="6" name="data">data/data1.dat</dataset>\\
\hspace*{0.5cm}</group>\\
    </group>\\
\end{kod}

\subsubsection*{Prepare HDF5 file directly from raw file}

It has the same parameters as descriptor preparation mode, but output file name
has to end with \texttt{.h5}. For volumes \texttt{-o} parameter can be skipped,
application will convert file directly to HDF5 using default name, which will be
ODIM name when proper parameter will be provided in \texttt{options.xml} file or
standard date-format name otherwise.

Examples of use:
\begin{kod}
\$ odim -i input.ppi -o ppi.h5 -p RAINBOW -f IMAGE\\ \\
\$ odim -i input.vol -p RAINBOW -f PVOL
\end{kod}

\subsubsection*{Prepare Rainbow volume file from HDF5 file}

Input filename has to have a \texttt{.h5} or \texttt{.hdf} extension, and output
filename has to have \texttt{.vol} extension.

For example:
\begin{kod}
\$ odim -i input.h5 -o output.vol -p RAINBOW -f PVOL

\end{kod}

\subsection{Baltrad Feeder}

\texttt{OdimH5} allows users to send HDF5 files into \texttt{BaltradDex} system.
To send a file use following command:
\begin{kod}
\$ odim -i input.h5 -r Brzuchania -s IMGW.pl
-a~http://172.30.9.34:8084/BaltradDex/dispatch.htm
\end{kod}

It sends a single HDF5 file to the server but it can work as a continuous
\texttt{Baltrad} feeder as well with online conversion to HDF5 format. It works
automatically with specific options provided by user.

To run feeder use \texttt{-c} option and provide \texttt{Baltrad} details in
\texttt{options.xml} in \texttt{<baltrad>} section.

Example of use:

\begin{kod}
\$ odim -c
\end{kod}

\begin{kod}
<?xml version="1.0" ?>\\
<!-- ODIMH5 options -->\\
<options>\\
\hspace*{0.5cm}<radar name="SWI">\\
\hspace*{1cm}<WMO\_id>WMO:12220,NOD:plswi,PLC:Swidwin</WMO\_id>\\
\hspace*{1cm}<file\_name>T\_PAGZ46\_C\_SOWR\_</file\_name>\\
\hspace*{1cm}<directory>rb5input/SWI</directory>\\
\hspace*{0.5cm}</radar>\\
\hspace*{0.5cm}<baltrad>\\
\hspace*{1cm}<server>http://172.31.50.150:8084/BaltradDex/dispatch.htm</server>\\
\hspace*{1cm}<sender>rainbow.imgw.pl</sender>\\
\hspace*{0.5cm}</baltrad>\\
</options>\\
\end{kod}
\subsection{Sending file by FTP}

\texttt{OdimH5} allows users to send HDF5 file using FTP. It works similar to
\texttt{Baltrad} feeder. To run FTP feeder use \texttt{-c} option and provide
\texttt{FTP} details in \texttt{options.xml} in \texttt{<ftp>} section.

\begin{kod} 
\hspace*{0.5cm}<ftp>\\
\hspace*{1cm}<radars>SWI POZ</radars>\\
\hspace*{1cm}<address>ftpops.metoffice.gov.uk</address>\\
\hspace*{1cm}<login>***</login>\\
\hspace*{1cm}<password>***</password>\\
\hspace*{1cm}<directory></directory>\\
\hspace*{1cm}<subfolders>0</subfolders>\\
\hspace*{0.5cm}</ftp>\\
\end{kod}

\subsection{Help}

To display help menu in program use following parameter:
\begin{kod}
\$ odim -h
\end{kod}

\section{Development support and guidelines}

This section of document provides information of particular interest to
developers of OdimH5 application.

To use OdimH5 API for creating HDF5 file the following classes have to be
imported:
\begin{kod}
import ncsa.hdf.hdf5lib.HDF5Constants;\\
import pl.imgw.odimH5.model.HDF5Model;
\end{kod}

\subsection{Creating new HDF5 file}

To create new file use {\bf H5Fcreate\_wrap} method from {\bf HDF5Model} class:
\begin{kod}
public {\bf int H5Fcreate\_wrap}({\bf String} filename, {\bf int}
access\_mode,\\ \hspace*{0.5cm}{\bf int} create\_id, {\bf int} access\_id, {\bf boolean}
verbose)
\end{kod}
This is helper method creating new HDF5 file.\\
Parameters:\\
\hspace*{0.5cm}{\bf filename} File name\\
\hspace*{0.5cm}{\bf access\_mode} File create mode\\
\hspace*{0.5cm}{\bf create\_id} Create identifier\\
\hspace*{0.5cm}{\bf access\_id} Access identifier\\
\hspace*{0.5cm}{\bf verbose} Verbose mode toggle\\
Returns:\\
\hspace*{0.5cm}File identifier\\

Example of use:
\begin{kod}
{\bf HDF5Model} proc = new HDF5Model();\\
{\bf int} file\_id = proc.H5Fcreate\_wrap(outputFileName,\\
\hspace*{0.5cm}HDF5Constants.H5F\_ACC\_TRUNC, HDF5Constants.H5P\_DEFAULT,\\
\hspace*{0.5cm}HDF5Constants.H5P\_DEFAULT, {\bf true});
\end{kod}

\subsection{Creating new HDF5 group}

To create new group use {\bf H5Gcreate\_wrap} method from {\bf HDF5Model} class:
\begin{kod}
public {\bf int} H5Gcreate\_wrap({\bf int} file\_id, {\bf String} name, {\bf
int} size\_h{\bf int},\\
\hspace*{0.5cm} {\bf boolean} verbose)
\end{kod}
This is helper method creating new HDF5 group.\\
Parameters:\\
\hspace*{0.5cm}{\bf file\_id} File identifier\\
\hspace*{0.5cm}{\bf name} Group name\\
\hspace*{0.5cm}{\bf size\_hint} Size hint\\
\hspace*{0.5cm}{\bf verbose} Verbose mode toggle\\
Returns:\\
\hspace*{0.5cm}Group identifier\\

Example of use:
\begin{kod}
{\bf HDF5Model} proc = new HDF5Model();\\
{\bf int} child\_group\_id = proc.H5Gcreate\_wrap(file\_id, "/what", 0,\\
\hspace*{0.5cm} {\bf true});
\end{kod}

\subsection{Creating new HDF5 attribute}

To create new attribute use {\bf H5Acreate\_any\_wrap} method from {\bf
HDF5Model} class:
\begin{kod}
public {\bf void H5Acreate\_any\_wrap}({\bf int} group\_id, {\bf String} attr\_name,\\
\hspace*{0.5cm}{\bf String} attr\_class, {\bf String} attr\_value, {\bf boolean}
            verbose)
\end{kod}
This is helper method for creating and writing HDF5 attribute of given type.\\
Parameters:\\
\hspace*{0.5cm}{\bf group\_id} Group identifier\\
\hspace*{0.5cm}{\bf attr\_name} HDF5 attribute name\\
\hspace*{0.5cm}{\bf attr\_class} HDF5 attribute data type\\
\hspace*{0.5cm}{\bf attr\_value} Access identifier\\
\hspace*{0.5cm}{\bf verbose} Verbose mode toggle\\

Example of use:
\begin{kod}
{\bf HDF5Model} proc = new HDF5Model();\\
proc.H5Acreate\_any\_wrap(child\_group\_id, "object", "string",\\
\hspace*{0.5cm}"PVOL", {\bf true});\\
proc.H5Acreate\_any\_wrap(child\_group\_id, "nrays", "long",\\
\hspace*{0.5cm} "360", {\bf true});
\end{kod}

\subsection{Creating new HDF5 simple dataspace}

To create new simple dataspace use {\bf H5Screate\_simple\_wrap} method from
{\bf HDF5Model} class:
\begin{kod}
public {\bf int} H5Screate\_simple\_wrap({\bf int} rank, {\bf int} dim\_x, {\bf int} dim\_y,\\
\hspace*{0.5cm} {\bf long} maxdims[], {\bf boolean} verbose)
\end{kod}
This is helper method creating new HDF5 simple dataspace.\\
Parameters:\\
\hspace*{0.5cm}{\bf rank} Rank\\
\hspace*{0.5cm}{\bf dims} Dataspace dimensions\\
\hspace*{0.5cm}{\bf maxdims} Maximum dataspace dimensions\\
\hspace*{0.5cm}{\bf dim\_x} Dataspace x dimension\\
\hspace*{0.5cm}{\bf dim\_y} Dataspace y dimension\\
\hspace*{0.5cm}{\bf verbose} Verbose mode toggle\\
Returns:\\
\hspace*{0.5cm}Dataspace identifier\\

Example of use:
\begin{kod}
{\bf HDF5Model} proc = new HDF5Model();\\
{\bf int} dataspace\_id = proc.H5Screate\_simple\_wrap(2, 360, 250, {\bf null},
\hspace*{0.5cm}{\bf true});
\end{kod}

\subsection{Creating new HDF5 dataset}

To create new dataset use {\bf H5Dcreate\_wrap} method from {\bf
HDF5Model} class:
\begin{kod}
public {\bf int} H5Dcreate\_wrap({\bf int} file\_id, {\bf String} group\_name,\\
\hspace*{0.5cm}{\bf int} datatype\_id, {\bf int} dataspace\_id, {\bf long[]}
chunk, {\bf int} gZipLevel,\\
\hspace*{0.5cm}{\bf boolean} verbose)
\end{kod}
This is helper method creating new HDF5 simple dataset.\\
Parameters:\\
\hspace*{0.5cm}{\bf file\_id} File identifier\\
\hspace*{0.5cm}{\bf group\_name} Group name\\
\hspace*{0.5cm}{\bf datatype\_id} Datatype identifier\\
\hspace*{0.5cm}{\bf dataspace\_id} Dataspace identifier\\
\hspace*{0.5cm}{\bf chunk} A 2-D array containing the size of each chunk\\
\hspace*{0.5cm}{\bf gZipLevel} Gzip compression level\\
\hspace*{0.5cm}{\bf verbose} Verbose mode toggle\\
Returns:\\
\hspace*{0.5cm}Dataset identifier\\

Example of use:
\begin{kod}
{\bf HDF5Model} proc = new HDF5Model();\\
{\bf long}chunk[] = new long[2];\\
chunk[0] = rays;\\
chunk[1] = bins;\\
{\bf int} grandgrandchild\_group\_id =
proc.H5Dcreate\_wrap(grandchild\_group\_id,\\ "data",
HDF5Constants.H5T\_STD\_U8BE, dataspace\_id, chunk, 6, verbose);
\end{kod}

\subsection{Writing HDF5 dataset}

To write HDF5 dataset use following method:

\begin{kod}
{\bf public int} H5Dwrite\_wrap({\bf int} dataset\_id, {\bf int}
mem\_type\_id,\\
\hspace*{0.5cm}{\bf int} mem\_space\_id, {\bf int} file\_space\_id, {\bf int}
xfer\_plist\_id, {\bf Object} buf,\\
\hspace*{0.5cm}{\bf boolean} verbose)
\end{kod}
Helper method for writing HDF5 dataset.\\
Parameters:\\
	\hspace*{0.5cm}{\bf dataset\_id} Dataset identifier\\
    \hspace*{0.5cm}{\bf mem\_type\_id} Memory type identifier\\
    \hspace*{0.5cm}{\bf mem\_space\_id} Memory space identifier\\
    \hspace*{0.5cm}{\bf file\_space\_id} File space identifier\\
    \hspace*{0.5cm}{\bf xfer\_plist\_id} \\
    \hspace*{0.5cm}{\bf buf} Data buffer\\
    \hspace*{0.5cm}{\bf verbose} Verbose mode toggle\\
Returns:\\
    \hspace*{0.5cm}Operation status\\

Example of use:
\begin{kod}
{\bf HDF5Model} proc = new HDF5Model();\\
proc.H5Dwrite\_wrap(grandgrandchild\_group\_id,\\
\hspace*{0.5cm}HDF5Constants.H5T\_NATIVE\_INT, HDF5Constants.H5S\_ALL,\\
\hspace*{0.5cm}HDF5Constants.H5S\_ALL, HDF5Constants.H5P\_DEFAULT,\\
\hspace*{0.5cm}infDataBuff, verbose);\\
\end{kod}

\section{Troubleshooting}

To report a bug please send information to
\href{mailto:lukasz.wojtas@imgw.pl}{maciej.szewczykowski@imgw.pl}

\section{Major Improvements and bug fixes}
 
Version 2.6 (Release date: 2010-06-10)
\begin{itemize}
    \item Added Baltrad feeder.
    \item Added direct HDF5 converter.
\end{itemize}
Version 2.11 (Release date: 2010-11-15)
\begin{itemize}
    \item Sending files by FTP.
    \item RAINBOW 5.31.1 format supported.
\end{itemize}
Version 2.12 (Release date: 2010-12-29)
\begin{itemize}
    \item HDF5 to Rainbow format conversion.
\end{itemize}
Version 2.14 (Release date: 2011-02-09)
\begin{itemize}
    \item Reading Quality Index from Rainbow volume files supported.
\end{itemize}
Version 2.14a (Release date: 2011-04-26)
\begin{itemize}
    \item New manual.
    \item Missing ``/Conventions'' tag added.
    \item File compression optimized.
\end{itemize}
Version 2.14b (Release date: 2011-06-07)
\begin{itemize}
    \item 32- and 64-bit platform bug fixed.
\end{itemize}
Version 2.16 (Release date: 2013-10-29)
\begin{itemize}
    \item velocity polar volumes conversion added
\end{itemize}
\end{document}
